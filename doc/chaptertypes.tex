% $Revision$
% $Header: E:/iwcvs/fc/doc/chaptertypes.tex,v 1.5 2008/05/05 10:04:15 iw Exp $
% $Log: chaptertypes.tex,v $
% Revision 1.5  2008/05/05 10:04:15  iw
% - redesigne terminal/nonterminal issue
% - \sym and \term now produces same color
% - record type section move to chapter 4
% - chapter 4 continued
%
% Revision 1.4  2008/04/20 07:20:45  iw
% - more doku
%
% Revision 1.3  2008/03/28 12:40:52  iw
% - written some pages of documentation
%
% Revision 1.2  2007/10/01 16:16:59  iw
% - new chapters
%
% Revision 1.1  2007/09/22 16:01:19  iw
% - documentation
%
% Revision 1.4  2006/11/12 14:49:13  iw
% - implemented layout
%
% Revision 1.3  2006/10/09 17:23:25  iw
% 2 small crrections
%
% Revision 1.2  2006/07/12 20:30:52  iw
% explain new record syntax
%
% Revision 1.1  2006/06/29 21:14:26  iw
% initial revision (splitting up of Docu.tex)
%
%

\chapter{Types} \label{types}

\section{Native types}
\subsection{Standard native types}

A number of basic types are predefined in \frege{}. They
are all based on some type of the \java{} language. For an
overview see \autoref{standardnativetypes}.

\begin{figure}[hbt]
\fbox{
\begin{tabular}{l|l|p{0.3\textwidth}}
\frege{} type name & based on \java{} type & Remarks\\
\hline
Bool & boolean &\\
Char & char & \\
Int & int & \\
Long & long & \\
Float & float & \\
Double & double & \\
String & java.lang.String & \\
Regex & java.util.Pattern & \\
Matcher & java.util.Matcher & \java{} data structure for inspecting
results of pattern matches\\
Throwable & java.lang.Throwable &
see discussions about native functions \\
StringArr & java.lang.String\texttt{[]} \\
IntArr & int\texttt{[]} \\
Array elem & \textit{elem}\texttt{[]} & polymorphic array of \frege{}
values\\
\end{tabular}
}
\caption{Standard native \frege{} types}
\label{standardnativetypes}
\end{figure}

\subsection{User defined native types}

Any \java{} type may be introduced as \frege{} type. For example, the
following declaration \ex{data Url = native java.net.URL;} declares the
\frege{} type \term{Url}.


\section{Function types}

Function types are denoted
{\it t1} \arrow{} {\it t2},
where {\it t1} and
{\it t2} are types. A value with function type {\it t1}
\arrow {\it t2} can be applied to a
value with type {\it t1} and such an expression has type {\it
t2}. However, a function is a value like any other. It may be
stored in lists, or other functions may be applied to it.

\section{Algebraic data types}

%\subsection{Pre-defined algebraic types}

\subsection{Unit type} \label{unittype}

The unit type \sym{()} is predefined through the compiler. It has exactly one value, also named \sym{()}. The unit type is often the result type of impure functions that exist for their side effects.

\begin{code}
    // data () = ()      predefined by compiler
    // Prelude defines the following
    derive Eq ()
    derive Ord ()
    derive Show ()
\end{code}

\sym{()} is an instance of type classes Eq, Ord and Show.

\subsection{List type} \label{listtype} \index{list type}

One of the most used types in functional programming is the list type.
For this reason, the list type enjoys special syntactic support in the
\frege{} language.

\paragraph*{List type notation}

List types are denoted \ex{[{\it e}]} where {\it e} is the type of the
list elements. All elements of the list are of the same type.

\paragraph*{List constructors}

Let {\tt e} be a value of type {\it a} and {\tt es} be a value of type
{\it [a]}. Then \ex{e:es} is a (new) list of type {\it [a]} with head
{\tt e} and tail {\tt es}.

The list construction operator {\tt :} is right associative so that
\ex{e1:e2:es} is equivalent to \ex{e1:(e2:es)}

The construct {\tt []} denotes the empty list.

Finite lists can be written by listing all elements inside square
brackets. For example, the list
\ex{[1,2,3]}
is the same as
\ex{1:2:3:[]}

\paragraph*{List patterns}  \label{listpattern}

List patterns look similar to list constructors, except that they are
made of other patterns instead of values.

The pattern \ex{[]} matches the empty list.

Let \texttt{p} and \texttt{ps} be patterns.
Then the pattern \ex{p:ps} matches a list with at least one element, if
\texttt{p} matches the head of the list and \texttt{ps} matches the tail
of the list.
Since \texttt{ps} may be another list pattern, one can match arbitrary
many initial parts of a list: \ex{a:b:c:ns}
This would match a list with at least 3 elements.

In analogy to list constructors for finite lists, a pattern like
\ex{[p1, p2, p3]} matches a list with exactly 3 elements, since it's
equivalent to \ex{p1:p2:p3:[]}

\subsection{Tuple types} \label{tupletypes} \index{tuples}

Tuples are data structures that can hold a fixed number of values. The
components of a tuple may have different types.

\paragraph*{Tuple type notation}

Let $t_1, t_2, \cdots, t_n$ be types. Then
\texttt{(}$t_1$\texttt{,} $t_2$\texttt{)} denotes a pair type, where
the first element has type $t_1$ and the second element has type
$t_2$. Likewise \texttt{(}$t_1$\texttt{,} $t_2$\texttt{,}
$t_3$\texttt{)} denotes a 3-tuple type,
\texttt{(}$t_1$\texttt{,} $t_2$\texttt{,}
$t_3$\texttt{,} $t_4$\texttt{)} a 4-tuple type and so forth.

\paragraph*{Tuple constructors}

A list of $n$ $(2<=n<=26)$ expressions separated by commas and enclosed in
parentheses constructs a $n$-tuple.

\paragraph*{Tuple patterns} \label{tuplepattern}

A list of $n$ $(2<=n<=26)$ patterns separated by commas and enclosed in
parentheses is a pattern for matching a $n$-tuple.

\section{Type aliases}

\section{Record types}
\subsection{Record constructors}
Records may be created on
the fly, just like tuples:
\begin{code}
sqrt :: Int -> Float
let
  r = { a=sqrt 2, b=8, c='x', d=Nothing }
in ...
\end{code}


\section{Classes (Interfaces) and Instances}
\subsection{Pre-defined Classes and Instances}
